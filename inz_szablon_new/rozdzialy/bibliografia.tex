\begin{thebibliography}{99}
	% Zmieniæ na 99, w przypadku, gdy bibliografia liczy wiêcej ni¿ 9 pozycji.
	% W przypadku du¿ej liczby pozycji, wygodniej korzystaæ z BibTeXa.
	% Zalecany wzór wpisu bibliograficznego dla publikacji z czasopisma: nazwiska i inicja³y imion (przy dwóch imionach bez spacji po kropce!) autorów pisane s¹ kapitalikami. Nastêpnie dwukropek i tytu³ pracy pisany kursyw¹. Nastêpnie kropka i nazwa czasopisma, volume (pogrubiony), rok i numery stron. W przypadku podawania przedzia³ów liczbowych u¿ywa siê pó³pauzy (--) bez spacji, a nie ³¹cznika (-)!
	% Zalecany wzór wpisu dla ksi¹¿ki: nazwiska i inicja³y imion kapitalikami; nastêpnie dwukropek i tytu³ kursyw¹. Po kropce numer wydania (opcjonalnie), nazwa wydawnictwa, miejsce i rok wydania.
	\thispagestyle{fancy}
	\bibitem{yamada} \textsc{Yamada A., Iwane N., Harada Y., Nishimura S., Koyama Y., Tanaka I.}: \textit{Lithium Iron Borates as High-Capacity Battery Electrodes}. Advanced Materials \textbf{22} (2010) 3583--3587.
	\bibitem{pusheen} \texttt{http://www.pusheen.com/}, dostêp 1.02.2016 r.
	\bibitem{chemia-nieorg} \textsc{Durrant P.J., Durrant B.}: \textit{Zarys wspó³czesnej chemii nieorganicznej}. Wydanie pierwsze. Pañstwowe Wydawnictwa Naukowe, Warszawa 1965.
\end{thebibliography}